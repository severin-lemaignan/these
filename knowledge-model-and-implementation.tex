\chapter{ORO Implementation and integration in robots}
\label{chapter|implementation_integration}

\section{Some implementation notes}

This section presents some of the main technological choices that have
been made to implement the knowledge base.

Some of the main algorithms are presented here as well (like the algorithms for clasification and discrimination,~\ref{sect|discrinimation}).

\subsection{A centralized server-based implementation}
\label{sect|oro-serverbased}


\subsection{OWL-DL ontologies and Jena}
\label{sect|jena}

\subsection{Reasoning: the Pellet reasoner}
\label{sect|pellet}

\subsection{Classification and discrimination algorithms}
\label{sect|discrimination}

\section{Bindings to other components/languages}
\label{sect|interfacing}

%%%%%%%%%%%%%%%%%
\section{Integration in the robot architecture}

\subsection{RPC and events-oriented interactions}

\subsection{Acquiring Knowledge}

\subsubsection{Knowledge acquisition and modalities merging}

\paragraph{Perception}
\paragraph{Interaction}
\paragraph{External sources (Web, upper ontologies, ...)}
\paragraph{Learning}

\subsubsection{Perspective taking}
\label{sect|perspectivetaking}

\subsubsection{Grounding/anchoring strategies}

\subsubsection{Ability to automatically create new object instances}

%%%%%%%%%%%%%%%%%

\subsection{Integration with symbolic task planning and executive layers}

\fxnote{CRAM, SHARY, pyrobots...}

\subsection{Integration with natural language processors}


\subsubsection{Monitoring and debugging}

\subsubsection{Is it fast enough? Scalability and responsiveness}


