\chapter{Symbolic knowledge representation}

\fxnote{Support material: \emph{What is a knowledge representation} by Davis,
Shrobe and Szolovits,
\url{http://groups.csail.mit.edu/medg/ftp/psz/k-rep.html}}

\section{Preliminary considerations on knowledge}
\label{sect|on-knowledge}

First, we want to represent knowledge and not only information.  Both knowledge
and information are contextualized. Knowledge also has a \emph{cultural}
context that enable interpretation. 

Note that the origins of AI where in purely symbolic models, that were
not usable either. Rich interleaving between geometric reasoning and symbolic
models is required. 


\section{A typology of features of KRS for robotics}
\label{sect|krs-features}


This section proposes a more formal typology of desirable features for a such a
system. For each feature, we provide a short definition along with links to
relevant literature.

\fxnote{Should we also mention the general dynamics of the knowledge
representation system? ie, the system from the point of view of an on-going
cognitive process. See \cite{Henderson2011}}

\subsection{Expressiveness: What Can be Represented?}
\label{sect|expressiveness}

\subsubsection{Introduction: main logic formalisms}

The main role of a knowledge representation system is to provide an adequate
representation system to store facts and concepts that can be informally
described in natural language.

Formal logic aims at providing such a representation system with the added
value of providing a tractable support for inference and reasoning.

Most (but not all) of the systems we survey rely on a particular logic
formalism. The choice of the formalism has a strong impact, on one side, on the
range of ideas that can be expressed conveniently (\emph{practical
expressiveness}) or at all (\emph{theoretical expressiveness}), on the other
side, on the ability to solve the inference problem (called
\emph{decidability}: is a given logical sentence true in my model?) in a
tractable manner.

A large number of logic formalism do exist, we shall summarize below the most
relevant for systems actually deployed in current robotic architectures.

\emph{Predicate logic} is the family of logic formalisms the most commonly
found in knowledge representation. It distinguishes itself from the simpler
\emph{propositional logic} by the use of quantification to increase generality.
\emph{First-order logic} (FOL) is the subpart of \emph{predicate logic} where the
objects of \emph{predicates} (or \emph{formulae}) are simple \emph{terms},
while in \emph{higher-order logics}, predicates can be themselves objects of
other predicates.

The family of \emph{description logics}~\cite{Baader2008} also play an
important role. It is a subset of the first-order logic, with some extensions
in second-order logic. Description logics are notable because most of them are
known to be decidable. In description logic, axioms are build from
\emph{concepts}, \emph{roles} (that are unary or binary predicates) and
\emph{individuals}. The W3C OWL-DL standard is a widely-used language to describe
domain with the description logic.

\emph{Modal logics}, that allow for statement qualification
like \emph{possibility} or \emph{necessity}, have been shown to be closely
related to description logics.

\fxfatal{On modal logics, see the remark of McCarthy, in \cite{McCarthy2007}, section 3}

One last class of logics that is of particular relevance for robotic
applications is the \emph{probabilistic logics} or \emph{Bayesian logics}.
These logics provide a formal framework to reason on propositions whose truth
or falsity is uncertain. We elaborate below on the representation of uncertainty.


\paragraph{Some examples}

\emph{The robot knows that a blue bottle is laying on the table.}

\emph{The robot knows that the human knows about the position of the bottle,
but the robot does not know what the human actually know about it.}

\subsubsection{Open World and Close World Assumptions}

The \emph{close world} (CWA) vs. \emph{open world} (OWA) assumption names a
modelling choice on the \emph{completeness} of a knowledge domain. In the close
world assumption, a proposition that can not be proven true is assumed to be
false (\emph{negation by failure}), while in the open world assumption, a
proposition may be considered either true, false or unknown.

This distinction is important in robotics were the robot may have to manipulate
concepts with only partial knowledge on them. For instance, let imagine a robot
that sees a bottle on a table, whose bottom is hidden by another object. The
robot can not prove that the bottle is indeed \emph{on} the table. A knowledge
representation system relying on the closed world assumption would then assume
the bottle is \emph{not} on the table ($\lnot R^{CWA}_{isOn}(bottle, table)$)
whereas with the open world assumption, the proposition $R^{OWA}_{isOn}(bottle,
table)$ would be undecided. Example in table~\ref{table|cwa-owa-example} provides
a simple, concrete example of consequences of the CWA/OWA choice on reasoning.

\begin{table}
	\begin{center}
	\begin{tabular}{ll}
	{\bf Action} & {\bf Part involved} \\
	\hline
	{\tt PickSoftly} & hand \\
	{\tt PickAndPlace} & arm, hand \\
	{\tt MoveArm} & arm \\
	\hline
	\end{tabular}
	\end{center}
	\caption{Assuming the question is: \emph{select actions that do not require
	to move the arm}, a CWA reasoner would return {\tt PickSoftly} whereas an
	OWA reasoner would not return anything if the {\tt PickSoftly} action is
	not explicitly said not to involve the arm.}
	\label{table|cwa-owa-example}
\end{table}

Domains constrained with the closed world assumption lead to more tractable
inference problems, and allow for instance the use of logic languages like
Prolog. Thus, several approaches exists to \emph{locally close} a domain (\cf
Levesque~\cite{Levesque2008}, section 24.3.2 for a summary of those).


\subsubsection{Representation of uncertainty and likelihood}

Sources of uncertainty for a robot are two-fold: uncertainty \emph{intrinsic}
to facts (like \emph{``It may rain tomorrow''}), uncertainty caused by
imperfect perception of the world (\emph{``Is the bottle really on the
table?''}). Most logics do not account explicitly for uncertainty. It must be
either relied on specific logics (like Bayesian logics) or on extensions of
classical logics.

\subsubsection{Meta-cognition: knowledge on the knowledge}

Narrower, more technical dimension of the introspection.

%%%%%%%%%%
\subsection{How things are represented?}
\label{sect|higher-level-domain-representation}

\subsubsection{Role Representations}

Spatio-Temporal Representations:

\paragraph{Representation of time}

As an agent acting at human-like time scale and dealing with temporal concepts
(like actions), a robot may want to represent, and possibly to reason, about
time. Time representation is split into two distinct abilities: representing
time points (both in the past -- which is roughly equivalent to assignment of
timestamps to events the robot perceives -- and in the future), and
representing \emph{passing time} (durations, timespans) like in \emph{``the
eggs will be cooked in 10 min''}.

\fxfatal{Discuss time chronicles~\cite{Ghallab1996}}

We call a system that do not account for time (\ie that permanently lives in
present) \emph{atemporal}.

\paragraph{Representation of space}

\paragraph{Representation of events and actions}

\subsubsection{Context modeling}

\emph{Knowledge is contextualized information}\fxfatal{Find someone respectable
how said that :-)}: it is essential for the robot to associate the facts it
represents to a \emph{context}. The context carries the keys for the
interpretation of the information. It covers the \emph{domain of validity} of
the facts, the \emph{common-sense} knowledge required to fill the gaps in the
representation\fxfatal{give an example}, \fxfatal{What more?}.

\subsubsection{Possible-Worlds and representing what others know}
\label{sect|possible-worlds}

    
Linked to the context representation, but seen from another angle, knowledge
representation systems may provide explicit ways to model other point of view
on the world. This ability is often referred as the \emph{perspective taking}
ability.

\cite{Levesque2008}, p. 4

\subsubsection{Introspection: Who am I? What can I do?}
\label{sect|introspection}

\paragraph{The introspective robot}

\paragraph{Modelling of the robot capabilities}


%%%%%%%%%%
\subsection{Reasoning Techniques}
\label{sect|reasoning}

\subsubsection{Standard reasoning techniques}

Backtracking,...

\paragraph{Decidability}

...

\subsubsection{Lazy evaluation}
\label{sect|lazy-evaluation}


\subsubsection{Reasoning with uncertainty}


\subsubsection{(Non) Monotonic Reasoning}

\emph{Monotonic reasoning} means that addition of new assertions to a knowledge base
can only extend the set of assertions that can be inferred, while a
\emph{non-monotonic} reasoning scheme may lead to retraction of facts.
McCarthy coined a famous example to illustrate the need of non-monotonic reasoning:

\begin{quotation}
Consider putting an axiom in a common sense database asserting that birds can
fly. Clearly the axiom must be qualified in some way since penguins, dead birds
and birds whose feet are encased in concrete can't fly. A careful construction
of the axiom might succeed in including the exceptions of penguins and dead
birds, but clearly we can think up as many additional exceptions like birds
with their feet encased in concrete as we like. Formalized non-monotonic
reasoning provides a way of saying that a bird can fly unless there
is an abnormal circumstance and reasoning that only the abnormal circumstances
whose existence follows from the facts being taken into account will be
considered.
\end{quotation}

Another important application of non-monotonic reasoning is representation of
change: for example, to make an omelette, you need to crack eggs and wipe them.
The eggs disappear and are replaced by an omelette:

$Egg(a) \wedge Egg(b) \wedge MakeOmelette(a, b, c) \to \lnot Egg(a) \wedge
\lnot Egg(b) \wedge Omelette(c)$

The insertion of the proposition $MakeOmelette(a, b, c)$ leads to retraction of
other facts. This rule requires non-monotonic reasoning to be applied.

\emph{Default logic} is one of the formal logic that account for representing
general truth and exceptions to it. However, due to computational complexity of
these model (most of inferences in default logic are known to be $NP$-complete
problem), classical logics and most of the existing reasoners do not allow
non-monotonic reasoning. For instance, the SWRL rule language, usually
associated to the OWL-DL ontology language, do not allow non-monotonic
reasoning.

\fxfatal{Make clear 'who does not allow non-monotonic-reasoning': logics? rule
languages? reasoner?}

One important exception if the \emph{Negation as failure} inference rule, as
implemented by {\sc Prolog} for instance, that allows for non-monotonicity, but
only \emph{within the closed world assumption}.

\fxfatal{Give here an example of non-monotonic reasoning with Prolog}

A monotonic system does not theoretically allow for knowledge retractation,
which is an important issue in the robotic context where the world model is
likely to be often altered.  However it is a practical issue only if the
reasoning process is \emph{continuous} during the whole robot's activity
lifespan. It is often possible to stop the reasoner, alter the knowledge, and
restart the inference process on a new domain.
\fxfatal{Rephrase to emphasize that when new evidences appear, it is anyway often a
good idea to restart the reasoner.}

\fxfatal{Mention that the 'change of world' issue can also be dealt with appropriate
time representation.}
\fxfatal{Mention that probabilistic reasoning lead to implicit non-monotonic reasoning}


\subsubsection{Presupposition accommodation}
\label{sect|presupposition-accomodation}

\emph{Presupposition accommodation} is the ability for the system to
automatically create a context allowing to make sense of a proposition.

For instance, we can imagine a human telling a robot \emph{Please get me the
bottle that is behind you}. If the robot has not yet see what is behind it, it
needs to assume (and represents in its knowledge model) that a undefined bottle
can be found somewhere in the half of space behind it.

A knowledge representation system able to copte with presupposition
accommodation would be able to take into account this (usually under-defined)
information that is not grounded into perception for later inferences.

This ability to imagine a physically state of the world that is not actually
perceived can be seen as the converse of the grounding ability.

Note also that presupposition accommodation implies a bidirectional link of the
symbolic knowledge model with a geometric (or physical) model of the
environment. This article focuses on symbolic knowledge representation systems,
but we shall mention when a KRS explicitly provides support for presupposition
accommodation.

\subsubsection{Prediction, projection and diagnosis tasks}
\label{sect|prediction-projection}

Levesque~\cite{Levesque2008} distinguish two main tasks, the \emph{projection
task} and the \emph{legality task}.

\paragraph{Projection task}: determining whether or not some condition while
hold after a sequence of actions.

\paragraph{Legality task}: determining whether a sequence of action can be
performed starting in some initial state.

\paragraph{Diagnosis}: this corresponds to the ability to rewind on past events
in case of failure to provide possible explanation. This can be seen as the
temporal reverse of the projection task.

\subsubsection{Physics-based reasoning}
\label{sect|physics}

\subsubsection{Planning}
\label{sect|planning}

Making decision based on prediction

\subsubsection{Modifying the knowledge structure}

All systems allow to modify the ABox, not always possible to alter the TBox


\subsubsection{Learning}
\label{sect|learning}

%%%%%%%%%%%%%%%%%
\subsection{Acquiring Knowledge}

\subsubsection{Knowledge acquisition and modalities merging}
\label{sect|knowledge-acquisition}

\paragraph{Perception}
\paragraph{Interaction}
\paragraph{External sources (Web, upper ontologies, ...)}
\paragraph{Learning}

\subsubsection{Grounding/anchoring strategies}
\label{sect|grounding}

\subsubsection{Ability to automatically create new object instances}
\label{sect|new-instances}

%%%%%%%%%%%%%%%%%
\subsection{Practical Integration in Robotic Architectures}
\label{sect|integration-robot}

\subsubsection{Integration with sensori-motor layers}
\label{sect|integration-sensorimotor}

Ability to ``listen'' to the robot internal structures.

\subsubsection{Integration with executive layers}
\label{sect|integration-executive-layers}

\paragraph{Language integration}

\paragraph{Events}

\subsubsection{Monitoring and debugging}
\label{sect|debugging}

\subsubsection{Is it fast enough? Scalability and responsiveness}
\label{sect|scalability}

%%%%%%%%%%%%%%%%%%%%%%%%%%
\subsection{Knowledge instanciation}

How much knowledge is available? Which content? How big is the knowledge base?

\begin{itemize}
	\item  Which underlying knowledge (\emph{common-sense}, \emph{upper knowledge}\ldots{})
	\begin{itemize}
		\item  top-down approach?
	\end{itemize}

\end{itemize}

\section{State of the art in knowledge representation systems for robotics}
\label{sect|krs-survey}


