\chapter{Symbolic knowledge representation}

\fxnote{Support material: \emph{What is a knowledge representation} by Davis,
Shrobe and Szolovits,
\url{http://groups.csail.mit.edu/medg/ftp/psz/k-rep.html}}

\section{Preliminary considerations on knowledge}
\label{sect|on-knowledge}

First, we want to represent knowledge and not only information.  Both knowledge
and information are contextualized. Knowledge also has a \emph{cultural}
context that enable interpretation. 


\section{Which end for knowledge representation?}
\label{sect|krs-purpose}

\subsection{Limits of traditional representation systems for robotics}
\label{subssect|limits}

\subsection{Specific requirements of robotics}
\label{subssect|robotics-specifics}

\fxnote{Taken from: \url{http://homepages.laas.fr/slemaign/wiki/doku.php?id=krs_survey}}
\begin{itemize}
	\item Run on service robot (robots that interact with objects in a semantic-rich environment),
	\item Ground their knowledge in the physical world (physically embedded),
	\begin{itemize}
		\item Able to \emph{resolve} entities
		\item Able to automatically create new object instances
	\end{itemize}
	\item Are able to merge different knowledge modalities,
	\item Are endowed with on-line, dynamic reasoning (not just a static dictionary)
\end{itemize}

\subsection{Benefits of a symbolic knowledge representation system}
\label{subssect|krs-benefits}

\fxnote{Note that the origins of AI where in purely symbolic models, that were
not usable either. Rich interleaving between geometric reasoning and symbolic
models is required.} 

\section{Formalisms for symbolic representation}
\label{sect|formalisms}

\section{Evaluation of a symbolic representation }
\label{sect|krs-evaluation}

\fxnote{Taken from: \url{http://homepages.laas.fr/slemaign/wiki/doku.php?id=krs_survey}}

\subsection{Intrinsic features}
\label{sect|eval-intrinsic-features}

\begin{itemize}
	\item Expressiveness
	\begin{itemize}
		\item Which logic formalism (DL, 2nd order…)
		\item OWA/CWA
		\item Representation of uncertainty
		\item (non) monotonic reasoning
		\item General knowledge + exception to this knowledge (blue sky/white sky)
		\item Microtheories
		\item Lazy evaluation
		\item Presupposition accomodation (ability to represent facts that are only verbally hinted, but not grounded into perception)
	\end{itemize}
    \item Explicit modeling of context of knowledge / domain of validity
    \item Representation of change (“The pancake dough disappears into a pancake”)
    \item Support for learning
    \item Support for introspection?
    \item Support for prediction/projection?
\end{itemize}

\subsection{Integration in robotic architectures}
\label{sect|eval-integration-robotic-archi}

\begin{itemize}
	\item Knowledge acquisition possible…
	\begin{itemize}
		\item …through interaction? How?
		\item …through the Web? How?
		\item …through learning? How?
		\item …through perception? How?
	\end{itemize}

	\item Integration with executive layers
	\item Events
	\item Relation to task planning
	\item Scalability and responsiveness
\end{itemize}

\subsection{Knowledge model}
\label{sect|eval-knowledge-model}

Which underlying knowledge (common-sense, upper knowledge…)
top-down approach?

\section{State of the art in knowledge representation systems for robotics}
\label{sect|krs-survey}


