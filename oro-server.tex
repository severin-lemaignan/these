\chapter{oro-server, design \& use of a symbolic knowledge representation system for robotics}
\label{chapter|oroserver}

\fxnote{This chapter focuses on the \textbf{functional description} of
oro-server. It does not need to match the actual implementation.}

\section{Functional overview}
\label{sect|functional-overview}

\fxnote{Here, we must present a general block diagram with all main functions.}

\subsection{ Open World reasoning}
\label{sect|open-world-reasoning}

\subsection{Classification and discrimination}
\label{subssect|discrimination}

\subsection{Memory}
\label{subssect|memory}

\subsection{Presupposition accomodation}
\label{sect|presupposition-accomodation}

\fxerror{TDB: this should show that the system can be enriched with facts that
are not perceived, but teached to the robot. How to demonstrate that?}


\section{Multiple models, persepective-taking and false beliefs}
\label{subssect|alterite}

\subsection{Perspective taking}
\label{subssect|perspective-taking}

\subsection{Management of False beliefs and Theory Of Mind}
\label{sect|theory-of-mind}

\fxnote{Sally and Ann experiment}

\section{A knowledge-centric architecture}
\label{sect|knowledge-centric-architecture}

\subsection{Semantic events}
\label{sect|semantic-events}

\subsection{Knowledge producers}
\label{sect|producers}

\subsubsection{Generalities}
\label{subssect|producers-generalities}

\subsubsection{SPARK}
\label{subssect|spark}

\subsection{Knowledge consumers}
\label{sect|consumers}

\subsubsection{Generalities}
\label{subssect|consumers-generalities}

\subsubsection{HATP}
\label{subssect|hatp}

\subsubsection{Executive layer (CRAM, SHARY...)}
\label{subssect|supervision}
