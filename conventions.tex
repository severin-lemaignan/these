\chapter*{\centering \begin{normalsize}Conventions and Notations\end{normalsize}}

This thesis relies on several notations and specific writing conventions to
describe symbolic knowledge and logical relations.

Ontologies and excerpts of ontologies presented in the work are mostly written
in the W3C's OWL language. As a derivative of XML, it uses namespaces to
declare the scopes of concepts. The main namespaces that are used in this work
are {\tt owl:}, {\tt rdf:}, {\tt rdfs:} (respective namespaces and schema
namespace of the Web Ontology Language and the Resource Description Framework),
{\tt cyc:} (concepts defined in the {\sc OpenCyc} upper ontology) and {\tt
oro:} (concepts defined in our \emph{OpenRobots Common-Sense} ontology). For
readability, the namespaces will be omitted when they are not required for the
understanding.

The table below summarizes the terminology that is used in this work to discuss
knowledge representation questions. While these terms are generally not
strictly synonyms, we will use them interchangeabily when no confusion may
arise.

\begin{center}
\begin{tabular}{p{3.5cm}|p{3.5cm}|p{3.5cm}|p{3.5cm}}
\toprule
Entity, \par Element, \par Concept & Class (OWL), \par Concept (DL) & Relation, \par Property (OWL), Role (DL), \par (Binary) Predicate & Instance (OWL), Individual (DL) \\
\bottomrule

\end{tabular}
\end{center}

Description Logic terminology (noted DL above) for classes (\ie \emph{concept})
and relations (\ie \emph{role}) will be used only in the specific context of
Description Logic. In other cases, the term \emph{concept} is used as a general
term that encompasses \emph{classes}, \emph{properties} and \emph{instances} in
the OWL terminology.

Depending on the context, a logical \emph{statement} is either a declarative sentence or
the meaning of this sentence (in this case, it is a \emph{fact} or a
\emph{belief}). Statements are generally represented as
\emph{triples} $\langle subject, predicate, object \rangle$. Statements that are
explicitely added to a knowledge base are called \emph{assertions}.

Again, we may use interchangeabily the terms \emph{statement},
\emph{assertion}, \emph{fact}, \emph{belief}, \emph{triple} when no confusion
arise.

Single concepts are typeset with this font: \concept{concept}, while statements
are typeset in this way (when represented as triples): \stmt{subject predicate object}.

Relations between concepts and rules rely on logical connectors. The table
below presents the most important ones that are found in this thesis.

\begin{center}
\begin{tabular}{ll}
\toprule
$\sqcap$ & intersection or conjunction of classes \\
$\sqcup$ & union or disjunction of classes \\
$\forall$ & universal restriction \\
$\exists$ & existential restriction \\
$\equiv$ & class equivalence \\
$\emptyset$ & empty set \\
\midrule
$\land$ & logical AND \\
$\lor$ & logical OR \\
$\to$ & implication \\
\bottomrule
\end{tabular}
\end{center}

Finally, we punctually use the Manchester
Syntax\footnote{\url{http://www.w3.org/TR/owl2-manchester-syntax/}} to present
in a readable way complex class expressions.

\clearpage
