\chapter{Conclusion}
\label{chapter|conclusion}

\section{Discussion}
\label{sect|discussion}

\subsection{Modeling the Real World}
\label{modeling_real_world}

The main challenge we address in this work can be formulated as \emph{How to
model real-world interaction in a symbolic way, processable by the robot to
make decisions}. In the paper we used several times the term \emph{grounding} to
describe the process of binding percepts to symbols (later organized in a
first-order logic framework).  We would like to relate it to
Sloman's~\cite{Sloman2007} stance against the \emph{``Symbol Grounding meme''},
where he argues that symbolic grounding is bound to the representation of
somatic concepts (\ie roughly, the sensori-motor relationships that the robot
learns from its interaction with the world) which in turn severely constraints
the domain of concepts accessible to the robot. We could call this type of
grounding \emph{bottom-up} grounding, and Steels~\cite{Steels2007} claims it is
a solved issue.

For us, \emph{grounding} is on the contrary a \emph{top-down} activity: the
robot needs to automatically bind a representation (for instance, a word
uttered by a human, an image taken from a camera, a sentence extracted from
Wikipedia) to an unambiguous, context-dependent, internal concept. This 
concept may (or may not) be \textit{a priori} available to the robot as a
pre-loaded ontology (what we previously called the cultural background of the
robot).

Note also that perception issues have been solved in our experiments by using a tag-based object identification
method. In section~\ref{informational_content_extraction} we give an example
where the human says ``the yellow banana is big''. It is assumed in the example
that the robot already knows about a banana instance that is yellow. In our
experiments, this kind of knowledge was either hard coded in scenario-specific
ontologies (\eg \stmt{banana\_01 type Banana} where \concept{banana\_01}
is the id of the banana's tag) or taught to the robot with prescriptive
sentences like ``Learn that this is a banana'' while pointing at the banana's
tag. It would be interesting to extend this approach with automatic classifiers
(for colour, size, etc.). If the robot later discovers a yellowish and large
object, an utterance like ``the yellow banana is big'' could be used to assert
that this object is a banana.  A similar approach focused on the combination of visual perception and communication modalities to achieve visual learning has been 
developed by~\cite{Vrecko2009}.

While the examples we develop are all based on symbols that have a physical
meaning, the system deals equally well with abstract, \emph{exo-somatic},
concepts like \emph{Time}, \emph{Event} or \emph{Place}. Demonstrating this in
real experiments would be an interesting development.

Amongst the other shortcomings of our architecture, neither the \emph{domain of
validity} nor the context of a fact are represented in a satisfying way (we do
store some kind of context --the agent's mental model for instance). This
information is meta-information on the knowledge. While the ORO framework
allows them through \emph{statement reification}, it does not offer yet a
convenient way to store them. One obvious limitation that derives from the lack
of efficient meta-knowledge is the absence of knowledge history.  With ORO, the
robot always lives in the present.

Along the same lines, our current framework lacks a proper management of
uncertainty which is essential for real world environments. A probabilistic
layer should be added by attaching truth probabilities to statements, similar
to~\cite{Jain2009}.

\subsubsection{On Thematic Roles and Action Models}

The current implementation relies on a small, predefined set of action verbs that can
be recognized from natural language (section~\ref{processing_of_actions}).
This constraint does not come from the resolution algorithm itself, but
rather from the difficulty to automatically extract the thematic roles associated
to a verb. 
This could be improved by linking a symbolic task
planner to the \textsc{Dialogs} module to dynamically provide the list of
actions that the robot can process, \ie actions for which the robot can produce
a plan. Additionally, we could exploit on-line resources like
\textsc{VerbNet}~\cite{Kipper2008}, which provides a large machine-processable
lexicon of English verbs along with their thematic roles.

\subsection{Knowledge and Embodiement}

The three experiments that were presented in the paper all illustrate how the
robot makes use of its embodied nature to establish a meaningful communication
with a human. Mainly, because the robot and the human share the same physical
environment and they perceive each other, we are able to create a mutual
context.

Sloman, in~\cite{Sloman2009}, argues however that the strong focus on
embodiment in the robotics community has hindered progress towards natural
human-robot interaction. Our approach has hopefully made clear that, similar to
Beetz et al.~\cite{Beetz2010}, we do not consider embodiment \emph{per se} outside of
a broader symbolic system, \ie our architecture is not bound to the morphology or
the low-level sensori-motor capabilities of a specific agent. 

However, we can build a model of the ``human point of view'' because the robot
perceives the human, and is able to estimate, at least partially, what the
human perceives or not. We infer that a human focuses on some object because
he/she points at it, looks at it, and besides, the object is visible to him.
This relies on the embodied nature of the interaction. In turn, this allows us
to understand the meaning of sentences like ``Give me that''.

We hope that this contribution shows that considering embodiment as the most
challenging and fruitful characteristic of robotics in regards to the whole AI
community does not contradict with a formal, highly symbolic approach of the
representation and decision problems that arise in robotics. 

Let us conclude this article briefly reviewing and linking Roy's list of challenges
for human-robot dialogue with our current approach: 
\begin{itemize}

	\item While more modalities (especially, deictic gestures and social gazes)
	can be added, we have actually proposed a \emph{cross-modal
	representation system}.

	\item One of the main feature of the \textsc{Dialogs} module is its ability
	to interactively ground concepts through disambiguation, bringing the
	ability for the robot to \emph{associate words with perceptual and action
	categories}.

	\item The ORO knowledge base offers some support for the \emph{modeling of
	context}, but a lot remains to be done in this respect.

	\item \emph{Figuring out the right granularity of models} is partially
	solved by supporting both a geometric reasoning level and a purely symbolic
	level. Generally speaking, it appears that complex robotic systems need
	to operate with a dynamic granularity, depending on the task to achieve.

	\item \emph{Temporal modeling} is currently missing in our architecture,
	and symbolic and geometric \emph{planning} is accomplished outside of the
	knowledge representation loop we presented here. We see planning as an
	essential tool to build predictive knowledge, and we are looking into this
	direction.

	\item Since we provide no time management, our system is currently not able
	to \emph{match past (learned) experiences with the current interaction}.
	This ability is obviously a key step for general action recognition, and
	seems of particular importance for the robot to assess the state of the
	interaction with the human.

	\item Finally, Roy mentions \emph{the ability to take into account the
	human perspective}: this is probably our main contribution which we are now
	trying to develop even further towards psychology-inspired experiments.

\end{itemize}

\subsection{Learning}
\label{sect|discussion-learning}

\section{What next?}
\label{sect|perspectives}


