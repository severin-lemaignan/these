\chapter*{\centering \begin{normalsize}Abstract\end{normalsize}}
\begin{quotation}
\noindent % abstract text

With the rise of the so-called \emph{cognitive robotics}, the need of advanced
tools to store, manipulate, reason about the knowledge acquired by the robot
has been made clear. But storing and manipulating knowledge requires first to
understand what the knowledge \emph{itself} means to the robot and how to
represent it in a machine-processable way.

This work strives first at providing a systematic study of the knowledge
requirements of modern robotic applications in the context of service robotics
and human-robot interaction. What are the expressiveness requirement for a
robot? What are its needs in term of reasoning techniques? What are the
requirement on the robot's knowledge processing structure induced by other
cognitive functions like perception or decision making? We propose a novel
typology of desirable features for \emph{knowledge representation systems}
supported by an extensive review of existing tools in our community.

In a second part, the thesis presents in depth a particular instantiation of a
knowledge representation and manipulation system called \emph{ORO}, that has
been designed and implemented during the preparation of the thesis. We
elaborate on the inner working of this system, as well as its integration into
several complete robot control stacks. A particular focus is given to the
modelling of agent-dependent symbolic perspectives and their relations to
theories of mind.

The third part of the study is focused on the presentation of one important
application of knowledge representation systems in the human-robot interaction
context: situated dialogue. Our approach and associated algorithms leading to
the interactive grounding of unconstrained verbal communication are presented,
followed by several experiments that have taken place both at the {\it
Laboratoire d'Analyse et d'Architecture des Systèmes} at CNRS, Toulouse and at
the {\it Intelligent Autonomous System} group at Munich Technical University.

The thesis concludes on considerations regarding the viability and importance
of an explicit management of the agent's knowledge, along with a reflection on
the missing bricks in our research community on the way towards \emph{``human
level robots''}.

\end{quotation}
\clearpage
