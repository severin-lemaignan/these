\chapter*{\centering \begin{normalsize}Zusammenfassung\end{normalsize}}
\begin{quotation}
\noindent{\bf Verankerung der Interaktion: Wissensmanagement für interaktive Roboter}

\vspace{2em}

Mit dem Aufstieg der sogenannten kognitiven Robotik ist der Bedarf an
mächtigeren Werkzeugen gestiegen, um das Wissen vom Roboter zu speichern und
weiter zu verarbeiten.  Diese Arbeit stellt zuerst eine Studie über die
Anforderungen an solche Werkzeuge vor und schlägt eine neuartige Typologie von
wünschenswerten Eigenschaften für Wissensrepräsentations-Systeme vor.

Wir führen dann ein solches System namens ORO ein. Wir zeigen seine innere
Arbeitsweise sowie seine Integration in verschiedene Roboter-Architekturen. Ein
besonderer Fokus liegt auf Agenten Perspektiven und ihre Beziehungen zur
\emph{Theory of Mind}.

Der dritte Teil der Studie stellt eine Komponente zur Verarbeitung von Dialogen
vor, die die interaktive Verankerung der freien verbalen Kommunikation
ermöglicht. Wir schließen mit mehreren Experiment-Berichten und einer
Diskussion über die fehlenden Bausteine auf dem Weg zum \emph{``human level''}
Roboter.

\end{quotation}
\clearpage
