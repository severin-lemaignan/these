\chapter{{\tt oro-server} API}

This appendix lists the complete API of {\tt oro-server} as of version 0.8.

\paragraph{Base}
\begin{itemize}

    \item {\tt {\bf safeAdd}(Set)}: try to add news statements in long term
    memory, if they don't lead to inconsistencies (return false if at least one
    stmt wasn't added).

    \item {\tt {\bf safeAdd}(Set, String)}: try to add news statements with a
    specific memory profile, if they don't lead to inconsistencies (return
    false if at least one stmt wasn't added).

    \item {\tt {\bf check}(Set)}: checks that one or several statements are
    asserted or can be inferred from the ontology

    \item {\tt {\bf checkConsistency}()}: checks that the ontology is
    semantically consistent

    \item {\tt {\bf checkConsistency}(Set)}: checks that a set of statements
    are consistent with the current model

    \item {\tt {\bf help}()}: returns a human-friendly list of available
    methods with their signatures and short descriptions.

    \item {\tt {\bf getLabel}(String)}: return the label of a concept, if
    available.

    \item {\tt {\bf lookup}(String)}: try to identify a concept from its id or
    label, and return it, along with its type (class, instance,
    object\_property, datatype\_property).

    \item {\tt {\bf lookup}(String, String)}: try to identify a concept from
    its id or label and its type (class, instance, object\_property,
    datatype\_property).

    \item {\tt {\bf revise}(Set, String)}: 

    \item {\tt {\bf add}(Set)}: adds one or several statements (triplets S-P-O)
    to the robot model, in long term memory.

    \item {\tt {\bf add}(Set, String)}: adds one or several statements
    (triplets S-P-O) to the robot model associated with a memory profile.

    \item {\tt {\bf clear}(Set)}: removes statements in the given set

    \item {\tt {\bf remove}(Set)}: removes one or several statements (triplets
    S-P-O) from the ontology.

    \item {\tt {\bf update}(Set)}: update the value of a functional property.

\end{itemize}

\paragraph{Agents}
\begin{itemize}

    \item {\tt {\bf checkConsistencyForAgent}(String)}: check the consistency
    of a specific agent model.

    \item {\tt {\bf safeAddForAgent}(String, Set)}: try to add news statements
    to a specific agent model in long term memory, if they don't lead to
    inconsistencies (return false if at least one stmt wasn't added).

    \item {\tt {\bf safeAddForAgent}(String, Set, String)}: try to add news
    statements to a specific agent model with a specific memory profile, if
    they don't lead to inconsistencies (return false if at least one stmt
    wasn't added).

    \item {\tt {\bf discriminateForAgent}(String, Set)}: returns a list of
    properties that helps to differentiate individuals for a specific agent.

    \item {\tt {\bf findForAgent}(String, String, Set)}: tries to identify a
    resource given a set of partially defined statements in an specific agent
    model.

    \item {\tt {\bf findForAgent}(String, String, Set, Set)}: tries to identify
    a resource given a set of partially defined statements and restrictions in
    an specific agent model.

    \item {\tt {\bf getInfosForAgent}(String, String)}: returns the set of
    asserted and inferred statements whose the given node is part of. It
    represents the usages of a resource.

    \item {\tt {\bf listAgents}()}: returns the set of agents I'm aware of (ie,
    for whom I have a cognitive model).

    \item {\tt {\bf lookupForAgent}(String, String)}: lookup a concept in a
    specific agent model.

    \item {\tt {\bf addForAgent}(String, Set)}: adds one or several statements
    (triplets S-P-O) to a specific agent model, in long term memory.

    \item {\tt {\bf addForAgent}(String, Set, String)}: adds one or several
    statements (triplets S-P-O) to a specific agent model associated with a
    memory profile.

    \item {\tt {\bf clearForAgent}(String, Set)}: removes statements from a
    specific agent model.

    \item {\tt {\bf removeForAgent}(String, Set)}: removes one or several
    statements. Deprecated. Use clearForAgent instead.

    \item {\tt {\bf save}(String, String)}: exports the cognitive model of a
    given agent to an owl file. The provided path must be writable by the
    server.

    \item {\tt {\bf updateForAgent}(String, Set)}: updates one or several
    statements (triplets S-P-O) in a specific agent model, in long term memory.
\end{itemize}


\paragraph{Administration}
\begin{itemize}

    \item {\tt {\bf makeHtmlDoc}()}: returns a list of available methods in
    html format for inclusion in documentation.

    \item {\tt {\bf listMethods}()}: returns the list of available methods with
    their signatures and short descriptions as a map.

    \item {\tt {\bf stats}()}: returns some statistics on the server

    \item {\tt {\bf listSimpleMethods}()}: returns a raw list of available
    methods.

    \item {\tt {\bf reset}()}: reload the base ontologies, discarding all
    inserted of removed statements, in every models

    \item {\tt {\bf list}(String)}: lists on the serveur stdout all facts
    matching a given pattern.

    \item {\tt {\bf save}()}: exports the current ontology model to an owl
    file. The file will be saved to the current directory with an
    automaticallygenerated name.

    \item {\tt {\bf save}(String)}: exports the current ontology model to an
    owl file. The provided path must be writable by the server.
\end{itemize}

\paragraph{Concepts comparison}
\begin{itemize}

    \item {\tt {\bf discriminate}(Set)}: returns a list of properties that
    helps to differentiate individuals.

    \item {\tt {\bf getDifferences}(String, String)}: given two concepts,
    return the list of relevant differences (types, properties...) between
    these concepts.

    \item {\tt {\bf getSimilarities}(String, String)}: given two concepts,
    return the list of relevant similarities (types, properties...) between
    these concepts.
\end{itemize}

\paragraph{Events}
\begin{itemize}

    \item {\tt {\bf registerEventForAgent}(String, String, String, String,
    List)}: registers an event on a specific agent model. Expected parameters
    are: agent, type, triggering type, variable, event pattern.

    \item {\tt {\bf registerEventForAgent}(String, String, String, List)}:
    registers an event on a specific agent model. Expected parameters are:
    agent, type, triggering type, event pattern.

    \item {\tt {\bf registerEvent}(String, String, String, List)}: registers an
    event. Expected parameters are: type, triggering type, variable, event
    pattern.

    \item {\tt {\bf registerEvent}(String, String, List)}: registers an event.
    Expected parameters are: type, triggering type, event pattern.

    \item {\tt {\bf clearEvent}(String, String)}: remove one specific event
    from a specific model.

    \item {\tt {\bf clearEventsForAgent}(String)}: remove all events associated
    to a specific model.

    \item {\tt {\bf clearEvent}(String)}: remove one specific event from the
    main model.

    \item {\tt {\bf clearEvents}()}: remove all events associated to the main
    model.
\end{itemize}

\paragraph{Querying}
\begin{itemize}

    \item {\tt {\bf find}(String, Set)}: tries to identify a resource given a
    set of partially defined statements about this resource.

    \item {\tt {\bf find}(String, Set, Set)}: tries to identify a resource
    given a set of partially defined statements plus restrictions about this
    resource.

    \item {\tt {\bf getInfos}(String)}: returns the set of asserted and
    inferred statements whose the given node is part of. It represents the
    usages of a resource.

    \item {\tt {\bf query}(String, String)}: performs one sparql query on the
    ontology

    \item {\tt {\bf getResourceDetails}(String)}: returns a serialized
    ResourceDescription object that describe all the links of this resource
    with others resources (sub and superclasses, instances, properties, etc.).

    \item {\tt {\bf getResourceDetails}(String, String)}: returns a serialized
    ResourceDescription object that describe all the links of this resource
    with others resources (sub and superclasses, instances, properties, etc.).
    The second parameter specify the desired language (following rfc4646).
\end{itemize}

\paragraph{Taxonomy}
\begin{itemize}

    \item {\tt {\bf getClassesOf}(String)}: returns a map of {class name,
    label} (or {class name, class name without namespace} is no label is
    available) of asserted and inferred classes of a given individual.

    \item {\tt {\bf getDirectClassesOf}(String)}: returns a map of {class name,
    label} (or {class name, class name without namespace} is no label is
    available) of asserted and inferred direct classes of a given individual.

    \item {\tt {\bf getDirectInstancesOf}(String)}: returns a map of {instance
    name, label} (or {instance name, instance name without namespace} is no
    label is available) of asserted and inferred direct instances of a given
    class.

    \item {\tt {\bf getDirectSubclassesOf}(String)}: returns a map of {class
    name, label} (or {class name, class name without namespace} is no label is
    available) of all asserted and inferred direct subclasses of a given class.

    \item {\tt {\bf getDirectSuperclassesOf}(String)}: returns a map of {class
    name, label} (or {class name, class name without namespace} is no label is
    available) of all asserted and inferred direct superclasses of a given
    class.

    \item {\tt {\bf getInstancesOf}(String)}: returns a map of {instance name,
    label} (or {instance name, instance name without namespace} is no label is
    available) of asserted and inferred instances of a given class.

    \item {\tt {\bf getSubclassesOf}(String)}: returns a map of {class name,
    label} (or {class name, class name without namespace} is no label is
    available) of all asserted and inferred subclasses of a given class.

    \item {\tt {\bf getSuperclassesOf}(String)}: returns a map of {class name,
    label} (or {class name, class name without namespace} is no label is
    available) of all asserted and inferred superclasses of a given class.
\end{itemize}

